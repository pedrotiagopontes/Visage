\chapter{Introdução} \label{chap:intro} TO-DO
NOTA:Versão antiga feita perto da antiga apresentação

\section{Contexto} \label{sec:context}

A sociedade atual caracteriza-se pelo constante fluxo de informação a que os seus indivíduos estão expostos. A título de exemplo e de acordo com dados divulgados pela Intel em \citep{IntelCorporation}, a cada minuto, na Internet são visualizados no Youtube 1,3 milhões de vídeos e carregadas para os seus servidores 30 novas horas de vídeo. Já o Flicker, no mesmo período de tempo, regista a visualização de 20 milhões de fotos e o carregamento de 3000 novas imagens para a sua plataforma.
Segundo os mesmos dados, atualmente o número de dispositivos com ligação à Internet é equivalente ao da população mundial e é esperado que em 2015 seja o dobro dessa população.
Este aumento e diversificação de conteúdos digitais e respetivos dispositivos que a eles acedem, torna particularmente difícil a sua categorização, organização e posterior pesquisa por porte dos utilizadores. A área da  Recuperação de Informação Multimédia (RIM), que trata a pesquisa de informação em todas as suas formas, ganha assim especial importância na atualidade, nomeadamente através da utilização de métodos baseados em conteúdo na pesquisa de informação, os quais tem a potencialidade de melhorar a exatidão da informação extraída, mesmo quando existem anotações textuais acerca dos conteúdos, ao possibilitar uma visão adicional sobre as coleções multimédia.\citep{Lew2006}.

O reconhecimento facial é um componente importante da capacidade de perceção humana, sendo efetuado de uma forma rotineira e quase sem esforço por parte dos seres humanos. Contudo, a construção de sistemas computacionais capazes de efetuar este tipo de reconhecimento de uma forma semelhante aos humanos em termos de eficácia é ainda uma área de investigação em aberto. Este é um problema de tal forma desafiante e interessante que tem despertado a atenção de investigadores das mais diversas áreas, ao longo dos últimos 40 anos, tais como: psicologia, reconhecimento de padrões, redes neuronais, visão por computador e computação gráfica. \citep{Zhao2003}.

De uma forma geral o problema de reconhecimento facial pode ser formulado da seguinte forma: dada uma imagem pretende-se identificar ou verificar a presença de uma ou mais pessoas na cena, comparando-a com uma base de dados de faces previamente guardada. A solução deste problema envolve três passos fundamentais: Deteção e segmentação das faces presentes na imagem, extração dos atributos faciais e por fim identificação ou verificação da identidade da pessoa.\citep{Zhao2003}
	

\section{Motivação} \label{sec:motivation}
A pesquisa ao nível do reconhecimento facial é motivada não só pelos desafios inerentes ao processo de reconhecimento facial, mas também pelos inúmeras aplicações onde a identificação de indivíduos é necessária \citep{Li2011}. Esta é uma área de investigação em aberto e para a qual ainda existe uma grande margem de progressão, no entanto, tal como mencionado anteriormente, esta é uma tarefa que envolve um conjunto grande de subproblemas que devem ser abordados separadamente.

No âmbito desta dissertação o impacto de filtros visuais de abstração de informação no processo de reconhecimento facial será o principal foque de investigação. Este filtros permitem uma simplificação do conteúdo visual ao retirar informação redundante e destacando a apenas a informação essencial. Estes filtros são tradicionalmente utilizados por artistas para comunicar de uma forma mais eficaz a informação pretendida ao retirar a informação desnecessária de imagens foto-realistas \citep{Kyprianidis2009}. Estudos efetuados demonstraram que aplicação destes filtros no âmbito da abstração de imagens para a ilustração automática de texto têm a potencialidade de melhorar a informação retornada, assim como reduzir significativamente as necessidades de processamento e armazenamento \citep{Coelho2012}. Tendo em conta os resultados previamente obtidos, e uma vez que não existem estudos relativos à utilização destes filtros no processo de reconhecimento facial, torna-se pertinente o estudo do seu impacto no âmbito desta dissertação.

Falar da crescente importância na atualidade e porquê - aplicações comerciais

\section{Objetivos} \label{sec:objetivosintro}
O principal objetivos deste projeto visa o estudo do impacto de filtros visuais de informação de contexto no processo de reconhecimento facial.
Adicionalmente propõe-se também o desenvolvimento de um sistema de reconhecimento facial de personalidades que integre a abstração de imagens no processo de reconhecimento facial. Este sistema deverá ser capaz de detetar as faces presentes numa imagem e apresentar uma lista de possíveis entidades nela contidas.

\section{Estrutura do Relatório} \label{sec:struct}
