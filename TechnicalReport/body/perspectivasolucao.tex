\chapter{Perspectiva de Solução} \label{chap:solução}

O reconhecimento facial em imagens sofreu uma evolução notável nos últimos 20 anos, tal como documentado no capítulo 1 deste relatório. Em cenários cooperativos, onde as condições de captura de imagens são controladas, considera-se mesmo que o problema de verificação 1:1 se encontra relativamente resolvido, uma vez que as taxas de reconhecimento atingidas são satisfatórias para a grande maioria das aplicações \citep{Li2011}. Existem também várias aplicações em situações reais com um bom nível de satisfação por parte dos seus utilizadores , como é o caso do sistema de fronteira automático dos aeroportos portugueses, ou o controlo de entradas nas cerimónias inaugurais dos jogos olímpicos de Pequim.

Cenários coope

Estudos efetuados demonstraram que aplicação destes filtros no âmbito da abstração de imagens para a ilustração automática de texto têm a potencialidade de melhorar a informação retornada, assim como reduzir significativamente as necessidades de processamento e armazenamento \citep{Coelho2012}. Tendo em conta os resultados previamente obtidos, e uma vez que não existem estudos relativos à utilização destes filtros no processo de reconhecimento facial, torna-se pertinente o estudo do seu impacto no âmbito desta dissertação.


A hipótese é que usar a abstração em imagens que vão ser ser alvo de reconhecimento facial pode melhorar a qualidade do reconhecimento.

\section{Objetivos} \label{sec:goals}
O principal objetivos deste projeto visa o estudo do impacto de filtros visuais de informação de contexto no processo de reconhecimento facial.
Adicionalmente propõe-se também o desenvolvimento de um sistema de reconhecimento facial de personalidades que integre a abstração de imagens no processo de reconhecimento facial. Este sistema deverá ser capaz de detetar as faces presentes numa imagem e apresentar uma lista de possíveis entidades nela contidas.

\section{Implementação} \label{sec:implementacao}
Uma vez que não se pretende neste projecto efectuar investigação ao nível dos algoritmos de reconhecimento facial, mas sim ao nível do impacto do uso de imagens abstraidas nesse sistemas, a implementação desses algoritmos terá por base a biblioteca OpenCV (Open Source Computer Vision) nomeadamente o módulo "Face Recognizer" (3 algorimos de reconhecimento facial - Eigenfaces, FisherFaces e Local Binary Patterns Histograms)

Ao nível dos filtros de abstração o estudo será efectuado utilizado o filtro Anisotropic Kuwahara Filter
Este filtro foi utilizado anteriormente, e com resultados positivos, em abstração de imagens para a recuperação de informação multimédia, pelo que se considera adequado a sua utilização no âmbito deste projecto.

Por último, ao nível da coleção de dados a analisar, e uma vez que este projeto se encontra a ser desenvolvido em parceria com o laboratório da Sapo da FEUP, temos em vista analisar a coleção de imagens Sapo Fama, a qual agrega uma base de dados de imagens de personalidades famosas nacionais e internacionais.

\subsection{OpenCV - Face Recognizer}
descrever genericamente 

\subsubsection{Eigenfaces}
lorem ipsum

\subsubsection{Fisherfaces}
lorem ipsum

\subsubsection{Local Binary Patterns Histograms}
lorem ipsum

\subsection{Base de dados}


\subsection{Avaliação Resultados}
	

\section{Resumo ou Conclusões}

Aliquam erat volutpat. Nunc pede ipsum, porttitor eu, bibendum non,
bibendum nec, nisl. Maecenas eget mauris. Nullam pulvinar. Curabitur
rutrum commodo est. Nam sapien pede, interdum eu, accumsan ultrices,
venenatis sit amet, tellus. Praesent ac ante bibendum enim varius
suscipit. Donec enim. Proin nisi. Quisque libero turpis, varius ut,
elementum vel, pulvinar sed, nunc. 
