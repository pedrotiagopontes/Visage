\chapter{Conclusão} \label{chap:conclusao}
O projeto de dissertação apresentado neste relatório técnico tem como objetivo estudar o impacto do uso de abstração de imagens no âmbito do reconhecimento facial automático. Para isso, pretendemos criar um sistema de reconhecimento facial de personalidades que integre a abstração de imagens no processo de reconhecimento facial. Este sistema deverá ser capaz de detetar as faces presentes numa imagem e apresentar uma lista de possíveis entidades nela contidas.

Através da análise do estado arte apresentada é possível concluir que o reconhecimento facial em imagens é um tema atual e onde se tem verificado um interesse crescente devido às suas múltiplas áreas de aplicação, assim como ao elevado valor comercial tradicionalmente associado a este tipo de soluções. 

O problema de reconhecimento facial é, no entanto, um problema complexo que integra um conjunto de sub-problemas  complexos. Estes sub-problemas, aliados as múltiplas áreas de aplicação do reconhecimento facial, fazem com que exista uma a grande variação do desempenho dos diversos sistemas existentes, a qual se encontra diretamente relacionada com as condições de utilização dos mesmos, nomeadamente ao no que diz respeito às galerias de imagens utilizadas. A este nível, em situações onde as condições de captura das imagens são controladas e existe uma cooperação ativa por parte dos utilizadores os resultados obtidos são muito satisfatórios, sendo mesmo considerado que o problema se encontra praticamente resolvido. Em contraste, em situações de captura não controladas e onde exista uma variação da iluminação, pose e expressão dos indivíduos este é ainda um problema desafiante e onde se verifica necessidade de investigação e melhoria dos resultados obtidos.

Os filtros de abstração de imagens constituem uma forma moderna e computacionalmente eficaz de abstração de informação, sendo tradicionalmente utilizados para comunicar mais eficazmente uma mensagem visual. O uso destes filtros para a pesquisa baseada em conteúdos com vista a ilustração automática de texto demonstrou resultados positivos, pelo que a abordagem apresentada neste relatório de utilização dos mesmos filtros no processo de reconhecimento facial tem a potencialidade de apresentar resultados igualmente positivos.

Os resultados obtidos no âmbito desta dissertação permitirão contribuir para o desenvolvimento da área através da utilização de uma nova abordagem ao processo de reconhecimento facial automático, assim como tirar conclusões relevantes acerca do uso destes filtros no processo de reconhecimento. Adicionalmente a implementação do sistema com base em uma biblioteca de código aberto permite também alargar o número de soluções atualmente existentes a este nível. Finalmente, o uso da biblioteca Sapo Fama, permitirá ainda uma valorização do projeto  do ponto de vista comercial, uma vez que esta pode ser uma das suas áreas diretas de aplicação.