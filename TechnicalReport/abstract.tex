\chapter*{Resumo}
%\addcontentsline{toc}{chapter}{Resumo}
O reconhecimento facial é efetuado de uma forma rotineira e quase sem esforço por parte das pessoas no seu dia-a-dia. No entanto, a construção de sistemas automáticos de reconhecimento facial é uma tarefa complexa que engloba um conjunto de sub-problemas específicos, nomeadamente a deteção e segmentação da face na imagem, a sua normalização ao nível da forma e iluminação e a extração das características distintivas da face, para que, por fim, seja efetuado o reconhecimento da identidade da pessoa representada. À resolução com sucesso deste conjunto de sub-problemas, colocam-se também um conjunto de desafios dos quais se destacam a variação ao nível da pose, iluminação e expressão das pessoas representadas.

Os desafios inerentes ao processo de reconhecimento, assim como, a vasta gama de aplicações onde a identificação de indivíduos é necessária, como por exemplo o controlo de acesso a informação, o reforço da segurança, o reforço da aplicação da lei, o entretenimento e a gestão de conteúdos multimédia, despoletou a atenção de inúmeros investigadores ao longo dos últimos 40 anos. Como consequência, verificou-se uma evolução notável ao nível da eficácia dos sistemas desenvolvidos, tal como pode ser verificado pelo registo de uma melhoria de duas ordens de grandeza desde 1993 nas últimas avaliações efetuadas. Em cenários cooperativos com condições de captura de imagens controladas, nomeadamente ao nível da pose, iluminação e expressões faciais, considera-se mesmo que o problema de reconhecimento facial se encontra praticamente resolvido. Por outro lado, em cenários não cooperativos esta é ainda uma área de investigação em aberto. Para isso, contribuí também o facto de o grau de satisfação dos sistemas utilizados estar diretamente relacionado com a sua área de aplicação e com os dispositivos através do qual é efetuado o reconhecimento, sendo que diferentes áreas e dispositivos exigem soluções diferentes e portanto investigação diferenciada.

Os filtros de abstração constituem uma forma moderna de simplificação do conteúdo visual, permitindo remover informação redundante e dar apenas destaque à mensagem visual a transmitir. O projeto Visage propõe o desenvolvimento de um sistema de reconhecimento facial de personalidades, onde seja utilizada a abstração de imagens no processo de reconhecimento, com o objetivo de analisar o impacto dos filtros de abstração nesse processo. Esse impacto deverá ser analisado não apenas ao nível da eficácia do reconhecimento efetuado, mas também ao nível das necessidades de processamento e armazenamento das imagens.

O sistema desenvolvido terá por base bibliotecas de código aberto que implementam algoritmos de reconhecimento facial próximos do estado da arte e o uso do filtro de abstração Kuwahara Anisotrópico. Ao nível das coleções de dados utilizadas, para além do recurso a uma biblioteca padrão de análise de sistemas de reconhecimento facial, temos em vista a utilização de uma coleção de imagens de figuras públicas famosas, nacionais e internacionais, que irá ser disponibilizada no âmbito da integração deste projeto no laboratório de investigação da Sapo da Universidade do Porto. 
Com base neste sistema será possível desenvolver aplicações de pesquisa visual capazes de encontrar fotos de uma personalidade específica ou apresentar, para uma foto de rosto fornecida pelo utilizador, a celebridade ou figura pública mais parecida.

\chapter*{Abstract}
%\addcontentsline{toc}{chapter}{Abstract}

Here goes the abstract written in English.

Lorem ipsum dolor sit amet, consectetuer adipiscing elit. Sed vehicula
lorem commodo dui. Fusce mollis feugiat elit. Cum sociis natoque
penatibus et magnis dis parturient montes, nascetur ridiculus
mus. Donec eu quam. Aenean consectetuer odio quis nisi. Fusce molestie
metus sed neque. Praesent nulla. Donec quis urna. Pellentesque
hendrerit vulputate nunc. Donec id eros et leo ullamcorper
placerat. Curabitur aliquam tellus et diam. 	

Ut tortor. Morbi eget elit. Maecenas nec risus. Sed ultricies. Sed
scelerisque libero faucibus sem. Nullam molestie leo quis
tellus. Donec ipsum. Nulla lobortis purus pharetra turpis. Nulla
laoreet, arcu nec hendrerit vulputate, tortor elit eleifend turpis, et
aliquam leo metus in dolor. Praesent sed nulla. Mauris ac augue. Cras
ac orci. Etiam sed urna eget nulla sodales venenatis. Donec faucibus
ante eget dui. Nam magna. Suspendisse sollicitudin est et mi. 

Fusce sed ipsum vel velit imperdiet dictum. Sed nisi purus, dapibus
ut, iaculis ac, placerat id, purus. Integer aliquet elementum
libero. Phasellus facilisis leo eget elit. Nullam nisi magna, ornare
at, aliquet et, porta id, odio. Sed volutpat tellus consectetuer
ligula. Phasellus turpis augue, malesuada et, placerat fringilla,
ornare nec, eros. Class aptent taciti sociosqu ad litora torquent per
conubia nostra, per inceptos himenaeos. Vivamus ornare quam nec sem
mattis vulputate. Nullam porta, diam nec porta mollis, orci leo
condimentum sapien, quis venenatis mi dolor a metus. Nullam
mollis. Aenean metus massa, pellentesque sit amet, sagittis eget,
tincidunt in, arcu. Vestibulum porta laoreet tortor. Nullam mollis
elit nec justo. In nulla ligula, pellentesque sit amet, consequat sed,
faucibus id, velit. Fusce purus. Quisque sagittis urna at quam. Ut eu
lacus. Maecenas tortor nibh, ultricies nec, vestibulum varius, egestas
id, sapien. 

Donec hendrerit. Vivamus suscipit egestas nibh. In ornare leo ut
massa. Donec nisi nisl, dignissim quis, faucibus a, bibendum ac,
diam. Nam adipiscing hendrerit mi. Morbi ac nulla. Nullam id est ac
nisi consectetuer commodo. Pellentesque aliquam massa sit amet
tellus. Vivamus sodales aliquam leo. 
