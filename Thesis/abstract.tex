\chapter*{Resumo}
%\addcontentsline{toc}{chapter}{Resumo}
O reconhecimento facial é efetuado de forma rotineira e quase sem esforço por parte das pessoas no seu dia-a-dia. No entanto, a construção de sistemas automáticos de reconhecimento facial é uma tarefa complexa que engloba um conjunto de sub-problemas específicos, nomeadamente a deteção e segmentação das faces presentes na imagem, a sua normalização e a extração das características distintivas das faces, para que, por fim, seja efetuado o reconhecimento da identidade das pessoas representadas. À resolução com sucesso deste conjunto de sub-problemas, colocam-se também um conjunto de desafios dos quais se destacam a variação ao nível da pose, iluminação e expressão das pessoas representadas.

Os desafios inerentes ao processo de reconhecimento, assim como a vasta gama de aplicações onde a identificação de indivíduos é necessária, como por exemplo o controlo de acesso a informação, a segurança, a aplicação da lei, o entretenimento e a gestão de conteúdos multimédia, despoletou a atenção de inúmeros investigadores ao longo dos últimos 40 anos. Como consequência, verificou-se uma evolução notável ao nível da eficácia dos sistemas desenvolvidos, considerando-se mesmo que o problema de reconhecimento facial em cenários cooperativos e com condições de captura de imagens controladas se encontra praticamente resolvido. Por outro lado, em cenários não cooperativos e onde se regista uma variação não controlada da captura das imagens, esta é ainda uma área de investigação em aberto. 

Os filtros de abstração constituem uma forma moderna de simplificação do conteúdo visual, permitindo remover informação redundante e dar destaque à mensagem visual a transmitir.

O projeto Visage visou o desenvolvimento de um sistema de reconhecimento facial de personalidades, baseado em código aberto, onde seja utilizada a abstração de imagens juntamente com um conjunto de outras tarefas de pré-processamento sobre as imagens a reconhecer. Com o sistema desenvolvido foi analisado o impacto dos filtros de abstração e das restantes tarefas de pré-processamento utilizadas no processo de reconhecimento. A avaliação de desempenho foi efetuada com recurso à biblioteca de imagens \textit{Labeled Faces in the Wild}, sobre duas perspetivas, \textit{Closed-set identification} e \textit{Image Retrieval}, e utilizando nove cadeias de pré-processamento de imagens distintas. Foram reportados para os resultados obtidos nos algoritmos \textit{Eigenfaces, Fisherfaces e Local Binary Patterns Histograms}.

Os resultados demonstram que a aplicação de filtros de abstração no processo de reconhecimento resulta num compromisso entre a diminuição dos requisitos de armazenamento das imagens e uma ligeira diminuição da eficácia da identificação. Ao nível das restantes tarefas de pré-processamento, a deteção e segmentação das faces presentes nas imagens revelou ser a fase de pré-processamento com maior importância para a obtenção de resultados positivos. Por último, dos três algoritmos analisados, o algoritmo \textit{Local Binary Patterns Histograms} revelou ter o melhor desempenho na maioria dos conjuntos analisados.

\chapter*{Abstract}
%\addcontentsline{toc}{chapter}{Abstract}
Face recognition is performed in a routine and effortless way by people in their daily life. However, building automated face recognition systems is a complex task that involves a set of specific sub-problems, namely face detection and segmentation, normalization of the extracted face and the extraction of the distinguishing features of the face, so that the recognition of the person present in a picture can be performed. In order to solve each of sub-problems with success, a set of challenges must be overcome, from which variation in pose, illumination and facial expression can be highlighted.

The challenges associated with the recognition process, as well as the wide range of areas where the identification of people is required, such as information access control, security, law enforcement, entertainment and multimedia content management, raised the attention of numerous researchers in the past 40 years. Consequently, a remarkable evolution was noticed in terms of system's performance. In cooperative scenarios with controlled image capturing conditions, it is considered that the problem of face recognition is almost solved. On the other hand,  in non-cooperative scenarios this is still an open research area.

Abstraction filters are a modern tools used to simplify visual content, allowing the removal of useless detail and the use of abstraction for effective communication.

The Visage project proposes the development of a face recognition system for public figures, based in open source libraries, where image abstraction along with some other pre-processing techniques are used in the process of recognition. The system has been used to analyse the impact of abstraction filters and the other pre-processing steps in face recognition. The performance was evaluated in Closed-set Identification and Image Retrieval perspectives, using the Labeled Faces in the Wild image library, with nine different pre-processing chains. Results were reported for Eigenfaces, Fisherfaces and Local Binary Patterns Histograms algorithms.

Results show that using abstraction filters in the process of automated face recognition results in a trade-off between reducing the images' storage requirements and a small decrease in the effectiveness of identification. Regarding the other pre-processing techniques, face detection and segmentation have shown the bigger positive impact on the results. Finally, three algorithms used in the analysis, Local Binary Patterns Histogram obtained the best performance in most of the analysed sets. 