\chapter*{Resumo}
%\addcontentsline{toc}{chapter}{Resumo}
O reconhecimento facial é efetuado de uma forma rotineira e quase sem esforço por parte das pessoas no seu dia-a-dia. No entanto, a construção de sistemas automáticos de reconhecimento facial é uma tarefa complexa que engloba um conjunto de sub-problemas específicos, nomeadamente, a deteção e segmentação das faces presentes na imagem, a sua normalização ao nível da forma e iluminação e a extração das características distintivas das faces, para que, por fim, seja efetuado o reconhecimento da identidade das pessoas representadas. À resolução com sucesso deste conjunto de sub-problemas, colocam-se também um conjunto de desafios dos quais se destacam a variação ao nível da pose, iluminação e expressão das pessoas representadas.

Os desafios inerentes ao processo de reconhecimento, assim como, a vasta gama de aplicações onde a identificação de indivíduos é necessária, como por exemplo o controlo de acesso a informação, o reforço da segurança, o reforço da aplicação da lei, o entretenimento e a gestão de conteúdos multimédia, despoletou a atenção de inúmeros investigadores ao longo dos últimos 40 anos. Como consequência, verificou-se uma evolução notável ao nível da eficácia dos sistemas desenvolvidos, tal como pode ser verificado pelo registo de uma melhoria de duas ordens de grandeza desde 1993 nas últimas avaliações efetuadas. Em cenários cooperativos com condições de captura de imagens controladas, nomeadamente ao nível da pose, iluminação e expressões faciais, considera-se mesmo que o problema de reconhecimento facial se encontra praticamente resolvido. Por outro lado, em cenários não cooperativos esta é ainda uma área de investigação em aberto. Para isso, contribui também o facto de o grau de satisfação dos utilizadores dos sistemas estar diretamente relacionado com a sua área de aplicação e com os dispositivos através dos quais é efetuado o reconhecimento, sendo que, diferentes áreas de aplicação e diferentes dispositivos exigem soluções diferentes e portanto investigação diferenciada.

Os filtros de abstração constituem uma forma moderna de simplificação do conteúdo visual, permitindo remover informação redundante e dar apenas destaque à mensagem visual a transmitir. O projeto Visage propõe o desenvolvimento de um sistema de reconhecimento facial de personalidades, onde seja utilizada a abstração de imagens no processo de reconhecimento, com o objetivo de analisar o impacto dos filtros de abstração nesse processo. Para além do impacto dos filtros de abstração pretende-se também analisar qual a importância de um conjunto de tarefas de pré-processamento efetuadas sobre as imagens no desempenho do sistema criado.

O sistema desenvolvido tem por base bibliotecas de código aberto que implementam algoritmos de reconhecimento facial próximos do estado da arte e o uso do filtro de abstração Kuwahara Anisotrópico. Com base neste sistema será possível desenvolver aplicações de pesquisa visual capazes de encontrar fotos de uma personalidade específica ou apresentar, para uma foto de rosto fornecida pelo utilizador, a celebridade ou figura pública mais parecida.

\chapter*{Abstract}
%\addcontentsline{toc}{chapter}{Abstract}
Face recognition is performed in a routine and effortless way by people in their daily life. However, building automated face recognition systems is a complex task that compromises a set of specific sub-problems, namely, face detection and segmentation, normalization of the extracted face in terms of light and shape and the extraction of the distinguishing features of the face, so that the recognition of the person present in a picture can be performed. In order to solve each of sub-problems with success, a set of challenges must be overcome, from which variation in pose, illumination and facial expression variation can be highlighted.

The challenges associated with the recognition process, as well as, the wide range of areas where the identification of people is required, such as information access control, security, law reinforcement, entertainment and multimedia content management, raised the attention of numerous researchers in the past 40 years. Consequently, a remarkable evolution was noticed in terms of system's performance, as can be seen by the two orders of magnitude improvement since 1993, registered in the last evaluation performed. In cooperative scenarios with controlled image capturing conditions, it is considered that the problem of face recognition is almost resolved. By the other hand,  in non-cooperative scenarios this is still an open research area. Nevertheless, the user satisfaction level of a system is directly related to its application area and the device with which recognition is being performed, whereas different areas of application and different devices require distinguished research.

Abstraction filters are a modern way to simplify visual content, allowing the removal of useless detail and the use of abstraction for effective communication. Visage project proposes the development of a face recognition system of public figures, where image abstraction is used in the process of recognition, with the aim of analyzing the impact of abstraction filters in face recognition. Furthermore, the impact in the performance of the system of a set of image pre-processing steps were also analyzed in the aim of this project.

The system developeded uses open source libraries that implement face recognition algorithms near to the state of the art and uses Kuwahara Anisotropic filter for image abstraction. With this system, the development of visual retrieval applications, capable of finding photos of public figures or, when a photo is provided by the user, finding the most similar celebrity to a person will be made available.