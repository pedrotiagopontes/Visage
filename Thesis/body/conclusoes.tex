\chapter{Conclusões e Perspetivas Futuras} \label{chap:conclusao}

\section{Conclusões}
A investigação levada a cabo no âmbito desta dissertação teve como principal objetivo estudar o impacto do uso da abstração de imagens no processo de reconhecimento facial automático, assim como, de um conjunto de tarefas de pré-processamento efetuadas sobre as imagens. Tendo em vista esse objetivo, foi desenvolvido o sistema de reconhecimento facial Visage. Para uma dada imagem fornecida ao sistema Visage, é aplicada sobre ela uma cadeia de pré-processamento, na qual a abstração de imagens se encontra incluída, e é efetuado posteriormente o seu reconhecimento, sendo devolvida uma lista ordenada de possíveis entidades contidas na imagem original.

Através da análise do estado arte apresentada é possível concluir que o reconhecimento facial em imagens é um tema atual e onde se tem verificado um interesse crescente devido às suas múltiplas áreas de aplicação, assim como ao elevado valor comercial tradicionalmente associado a este tipo de soluções. 

O problema de reconhecimento facial é, no entanto, um problema complexo que integra, também ele, um conjunto de sub-problemas  complexos. Estes sub-problemas, aliados as múltiplas áreas de aplicação do reconhecimento facial, fazem com que exista uma a grande variação do desempenho dos diversos sistemas existentes, a qual se encontra diretamente relacionada com as condições de utilização dos mesmos, nomeadamente ao no que diz respeito às galerias de imagens utilizadas. A este nível, em situações onde as condições de captura das imagens são controladas e existe uma cooperação ativa por parte dos utilizadores os resultados obtidos são muito satisfatórios, sendo mesmo considerado que, nestas situações, o problema se encontra praticamente resolvido. Em contraste, em situações de captura não controladas e onde exista uma variação da iluminação, pose e expressão dos indivíduos este é ainda um problema desafiante e onde se verifica necessidade de investigação na atualidade.

Por outro lado, os filtros de abstração de imagens constituem uma forma moderna e computacionalmente eficaz de abstração de informação, sendo tradicionalmente utilizados para comunicar mais eficazmente uma mensagem visual. Para além disso, o uso destes filtros para a pesquisa baseada em conteúdos com vista a ilustração automática de texto demonstrou resultados positivos, ao nível da informação retornada.

Tendo em conta a pertinência do estudo no âmbito do reconhecimento facial automático em situações de captura de imagens não controladas, assim como, a inexistência de estudos relativos ao impacto da utilização de filtros de abstração no processo de reconhecimento, a investigação levada a cabo no âmbito desta dissertação permite contribuir de forma relevante para o conhecimento existente na área. 

Ao nível das avaliações efetuadas, é possível concluir que a deteção e segmentação correta das faces constituem as etapas de pré-processamento mais relevantes para a obtenção de resultados positivos no reconhecimento dos indivíduos, independentemente do algoritmo de reconhecimento utilizado. Por outro lado, a normalização do contraste das imagens através da equalização do seu histograma, revela uma melhoria significativa nos resultados obtidos com particular ênfase no algoritmo \textit{Eigenfaces}. 

Finalmente, a integração da abstração de imagens no processo de reconhecimento apresenta um compromisso entre a diminuição da necessidade de processamento e armazenamento necessárias, com a ligeira diminuição da eficácia do reconhecimento. A este nível destaca-se o filtro kuwahara anisotrópico, o qual representa o maior grau de abstração dos filtros utilizados, permitindo uma diminuição considerável do tamanho da galeria processada, mas que regista também um maior impacto no desempenho do reconhecimento.

Por último, a implementação do sistema Visage com base em uma biblioteca de código aberto permite também alargar o número de soluções atualmente existentes a este nível, ao mesmo tempo que constitui uma base sólida para o desenvolvimento de futuras aplicações que tirem partido do reconhecimento facial automático em imagens no seu funcionamento.

\section{Perspetivas Futuras}
O sistema de reconhecimento facial desenvolvido, assim como as avaliações efetuadas, vão de encontro aos objetivos traçados no âmbito desta dissertação, permitindo contribuir ativamente para o conhecimento existente acerca da aplicação de filtros de abstração no processo de reconhecimento facial em imagens. De seguida, encontra-se resumido algum do trabalho futuro com vista a melhorar o sistema de reconhecimento facial Visage, assim como expandir as conclusões da avaliação efetuada.

A primeira etapa de pré-processamento aplicada no sistema Visage corresponde à deteção das faces presentes numa imagem. Nesta fase, é apenas considerada a existência de uma cara relevante na imagem e é utilizado um classificador em cascata treinado para caras em posição frontal. A expansão desta etapa de pré-processamento, permitindo a identificação de múltiplas pessoas na mesma imagem seria um próximo passo na melhoria da etapa de deteção facial. A utilização de vários classificadores diferentes correspondentes às múltiplas poses representadas nas diferentes imagens permitiria também efetuar uma deteção facial mais robusta, para além de possibilitar a criação de modelos de reconhecimento facial focados em reconhecer imagens de uma pessoa com uma pose específica.

Ao nível das restantes tarefas de pré-processamento aplicadas antes de efetuar a abstração das imagens existe também algum espaço para melhoria. Em primeiro lugar, o processo de alinhamento das imagens poderia ser melhorado de forma a torna-lo mais robusto e ter em consideração a pose de uma face no seu alinhamento. No que diz respeito às técnicas de normalização do contraste, a diferença obtida com recurso à equalização do histograma e com recurso à técnica \textit{CLAHE} indicam também que esta é uma área onde existe um espaço para melhoria.

A abstração de imagens foi efetuada com recurso a três filtros distintos, os quais representam três graus de abstração de complexidade diferente. No âmbito de futuras investigações a avaliação de outras técnicas de abstração de imagens poderá revelar-se relevante.

A arquitetura do sistema Visage baseou-se na implementação de múltiplas aplicações do tipo caixa preta, onde para um conjunto de dados de entrada é produzido um resultado especifico, sem que para isso seja necessário ter um conhecimento aprofundado acerca do funcionamento do sistema. A integração e eventual adaptação deste sistema a uma arquitetura cliente-servidor poderia permitir uma utilização dos métodos implementados de uma forma mais simples e intuitiva por parte de outras aplicações.


